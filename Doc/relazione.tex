\documentclass[10pt,a4paper,openany]{article}

\usepackage[english,italian]{babel} %lingue utilizzabili


%PER GRAFICA
\usepackage{graphicx}
\usepackage{fancyhdr} %per intestaz e piè di pagina
\usepackage{subfig}
\usepackage{float}


\usepackage[utf8]{inputenc} %codifica e input

\usepackage{hyperref} %indice linkato ai paragrafi di riferimento
\hypersetup{
  colorlinks,
  citecolor=gray,
  filecolor=brown,
  linkcolor=black,
  urlcolor=cyan
}

\usepackage{lastpage} %per sapere il numero di pagine totali

\usepackage{parcolumns} %testo affiancato in colonna

\usepackage{verbatim}%commenti a blocchi
\usepackage{listings} %Per inserire codice
\usepackage[usenames]{color} %Per permettere la colorazione dei caratteri 

\usepackage[a4paper,top=3cm,bottom=2cm,left=2cm,right=2cm] {geometry} %imposta pagina formato A4 e setta margini a piacimento
\usepackage[pdftex]{lscape} %per settare pagine in landscape

\renewcommand{\familydefault}{\sfdefault}   %mette tutto il carattere con uno stile ''grazioso''

\definecolor{editorGray}{rgb}{0.95, 0.95, 0.95}
\definecolor{editorOrange}{rgb}{1, 0.5, 0} % #FF7F00 -> rgb(239, 169, 0)
\definecolor{editorGreen}{rgb}{0, 0.6, 0} % #007C00 -> rgb(0, 124, 0)




%personalizzazione classe per il codice XML
\lstnewenvironment{xml}[1][] 
{\lstset{basicstyle=\scriptstyle \ttfamily, columns=fullflexible, keywordstyle=\color{blue}\bfseries, ndkeywordstyle=\color{blue}\bfseries , ndkeywords={references}, numberstyle=\color{red}, commentstyle=\color{editorGreen}, showstringspaces=false, stringstyle=\color{editorOrange},
language=XML, basicstyle=\small,
numbers=left, numberstyle=\tiny,
tabsize=2, stepnumber=10, numbersep=5pt, breaklines=true, frame=single, rulecolor=\color{black}, #1}}
{\lstset {language=XML,morekeywords={xml}}}{}


%personalizzazione classe per il codice HTML
\lstnewenvironment{html}[1][] 
{\lstset{basicstyle=\scriptstyle \ttfamily, columns=fullflexible,
keywordstyle=\color{blue}\bfseries, ndkeywords={content,=,charset=, id=, width=, height=},	 ndkeywordstyle=\color{editorGreen}\bfseries , numberstyle=\color{red} commentstyle=\color{red}, showstringspaces=false, stringstyle=\color{editorOrange},
language=HTML, basicstyle=\small,
numbers=left, numberstyle=\tiny,
tabsize=2, stepnumber=10, numbersep=5pt, breaklines=true, frame=single, rulecolor=\color{black}, #1}}{}

%personalizzazione classe per il codice C++
\lstnewenvironment{CPP}[1][] 
{\lstset{basicstyle=\scriptstyle \ttfamily, columns=fullflexible,
keywordstyle=\color{blue}\bfseries, ndkeywords={content,=,charset=, id=, width=, height=},	 ndkeywordstyle=\color{editorGreen}\bfseries , numberstyle=\color{red} commentstyle=\color{red}, showstringspaces=false, stringstyle=\color{editorGreen},
language=C++, basicstyle=\small,
numbers=left, numberstyle=\tiny,
tabsize=2, stepnumber=10, numbersep=5pt, breaklines=true, frame=single, rulecolor=\color{black}, #1}}{}


%%%%%%%%%%%%%%%%%%%%%%%%%%%%%%%%%%%%%%%%%%%%%%%%%%%%%%%%%%%%%%%%%%%%
%%%%%%%%%%%%%%%%%%%%%%%%%%%%%%%%%%%%%%%%%%%%%%%%%%%%%%%%%%%%%%%%%%%%
%%%%%%%%%%%%%%%%%%%%%%%%%%%%%%%%%%%%%%%%%%%%%%%%%%%%%%%%%%%%%%%%%%%%


\begin{document}

\begin{figure}
\centering
\includegraphics[angle=0,scale=.30]{LOGO.png}%%%%%%%%%%%%%%%%%%%%%%
\end{figure} 


\title{ \textbf{\huge PROGRAMMAZIONE AD OGGETTI}\vspace{0.7cm} \\ {\Huge Relazione del progetto} \vspace{0.3cm} \\ {\Large Anno accademico 2015/2016} }
%\date{}
 
\author{\textbf{Zecchin Giacomo (1070122)}}

\maketitle
\thispagestyle{empty}


%da qui stile intestazione e piè di pagina cosi definiti
\pagestyle{fancy}
\lhead{}
\rfoot{\vspace {1pt} Relazione del progetto}
\cfoot{\vspace{1pt} Pagina: \thepage\ di \pageref{LastPage}}
\fancyfoot[L]
	{\vspace{1pt}
   		\hyperref[sec:index] {\includegraphics[scale=0.06]{leftfoot.png}} %%%%%%%%%% icona linkata all'index
	 }
\renewcommand{\footrulewidth}{0.5pt}


\tableofcontents \label{sec:index} %CREA INDICE LINKABILE


\newpage


\section{Scopo del progetto}

Mediary è il risultato di questo progetto per il corso di programmazione ad oggetti che aveva come scopo quello di creare un applicativo sviluppato in C++/Qt.\\\\
L'applicazione intende rappresentare uno spazio personale nel quale annotare tutte le serieTv o i film già visti, preferiti oppure col desiderio di vedere in futuro.\\
Mediary consente quindi di accedere singolarmente come utente per aggiungere nuovi \textit{media} al proprio diario (\textit{diary}).\\
L'unione di queste due parole, \textbf{Media} e \textbf{Diary}, va difatti a creare il nome dell'applicazione.\\\\
Il progetto è semplice ma allo stesso tempo funzionale, diretto e comprensibile per l'utente; vediamo nel seguito tutte le funzionalità.


\section{Funzionalità offerte}

Avviata, l'applicazione presenta subito una finestra centrale per l'inserimento diretto dei dati necessari ad effettuare il \textbf{login} ed accedere  immediatamente all'area personale.\\
In alternativa, se non si possiedono già delle credenziali, è presente un bottone per la \textbf{registrazione} poco più in basso, che farà entrare nella finestra dedicata alla creazione di un nuovo utente.
In entrambi questi due casi sono presenti dei controlli di consistenza dei dati descritti più approfonditamente nella sezione *sez.* .\\
E' comunque data la possibilità di \textbf{modificare i propri dati} in un secondo momento.\\\\
Le funzionalità offerte per i media sono invece quelle di \textbf{creazione, modifica, visualizzazione per tipo ed eliminazione}. (in dettaglio qui *sez*).

\newpage

\section{Model}

Di seguito verrà descritta




\newpage

\section{View}

Il progetto è stato costruito rispettando il \textbf{pattern MVC}, separando quindi il \textit{Model} dalla \textit{View} con l'ausilio delle due classi che formano il \textit{Controller}.


\begin{comment}
In alto a destra è invece presente il bottone \textit{esci} per chiudere il programma.\\\\
Nel caso si eseguisse la procedura di l'autenticazione si verrebbe mandati alla finestra personale con la possibilità di aggiungere, modificare oppure eliminare media.

Nel mezzo è presente una tabella che permette di visualizzare, in ordine di inserimento dal più al meno recente, i media divisi per serieTv, film oppure tutti i tipi (l'opzione è selezionalibile tramite una comboBox).

E' possibile modificare un elemento cliccando al di sopra \textbf{titolo} del media che si intende manipolare, mentre per cancellarlo è necessario cliccare sopra l'icona del cestino nell'ultima colonna della tabella corrispondente alla riga del media (si aprirà un avviso per la conferma).

La stessa finestra offre inoltre la possibilità di modificare i propri dati personali cliccando nel bottone 
\end{comment}





%\selectlanguage{italian}	%inserirlo causerebbe errore in compilazione per lettere accentate-->non risolto


	
%per inserire un'immagine:

%\begin{landscape}
%\begin{figure}
%\subsection{Schema E-R iniziale del database}
%\centering
%\includegraphics[angle=0,scale=.40]{image.jpg}
%\hspace{1in}
%\label{schema}
%\caption{schema E-R}
%\end{figure}
%\end{landscape}



%inserire codice c++:

%\begin{CPP}
%#include <iostream>
%\end{CPP}







\end{document}
